\documentclass[oneside,13pt,a4paper]{article}

% Chargement d'extensions
\usepackage[utf8]{inputenc}
\usepackage[french]{babel}
\usepackage{graphicx}
\usepackage[top=3cm, bottom=3cm, left=3cm, right=3cm]{geometry}
\usepackage{amsmath}
\usepackage{amssymb}

% Liens et autres
\usepackage{hyperref}
\hypersetup{
    colorlinks=true,
    linkcolor=black,
	urlcolor=blue,
	pdftitle={Rendu},
	bookmarks=true,
}

% Bout de code
\usepackage{listings}
\usepackage{color}

\definecolor{mygreen}{rgb}{0,0.6,0}
\definecolor{mygray}{rgb}{0.5,0.5,0.5}
\definecolor{mymauve}{rgb}{0.58,0,0.82}

\lstset{
  backgroundcolor=\color{white},   % choose the background color; you must add \usepackage{color} or \usepackage{xcolor}; should come as last argument
  basicstyle=\footnotesize,        % the size of the fonts that are used for the code
  breakatwhitespace=false,         % sets if automatic breaks should only happen at whitespace
  breaklines=true,                 % sets automatic line breaking
  captionpos=b,                    % sets the caption-position to bottom
  commentstyle=\color{mygreen},    % comment style
  deletekeywords={...},            % if you want to delete keywords from the given language
  escapeinside={\%*}{*)},          % if you want to add LaTeX within your code
  extendedchars=true,              % lets you use non-ASCII characters; for 8-bits encodings only, does not work with UTF-8
  firstnumber=0,                   % start line enumeration with line 1000
  frame=single,	                   % adds a frame around the code
  keepspaces=true,                 % keeps spaces in text, useful for keeping indentation of code (possibly needs columns=flexible)
  keywordstyle=\color{blue},       % keyword style
  %language=C++,                    % the language of the code
  morekeywords={*,...},            % if you want to add more keywords to the set
  numbers=left,                    % where to put the line-numbers; possible values are (none, left, right)
  numbersep=5pt,                   % how far the line-numbers are from the code
  numberstyle=\tiny\color{mygray}, % the style that is used for the line-numbers
  rulecolor=\color{black},         % if not set, the frame-color may be changed on line-breaks within not-black text (e.g. comments (green here))
  showspaces=false,                % show spaces everywhere adding particular underscores; it overrides 'showstringspaces'
  showstringspaces=false,          % underline spaces within strings only
  showtabs=false,                  % show tabs within strings adding particular underscores
  stepnumber=1,                    % the step between two line-numbers. If it's 1, each line will be numbered
  stringstyle=\color{mymauve},     % string literal style
  tabsize=2,	                   % sets default tabsize to 2 spaces
}

% Commande pour notation 'NB :' (nota bene)
\newcommand\nb[1][0.3]{N\kern-#1emB : }

% csquotes va utiliser la langue définie dans babel
\usepackage[babel=true]{csquotes}

% pour afficher Schéma au lieu de figure dans les legende des images
\addto\captionsfrench{\def\figurename{Schéma}}

% Informations le titre, le(s) auteur(s), la date
\title{}
\author{

}
\date{\today}


\begin{document}
%\maketitle
\begin{titlepage}
	\centering
	{\scshape\LARGE Universite de Montpellier\par}
	{\scshape\Large\par}
	\vspace{1.5cm}
	{\huge\bfseries Structures, Protocoles, Objets IPC et Threads\par}
	\vspace{2cm}
	{\Large\itshape
		
		\par}

		\vspace{2cm}

	\begin{figure}[h]
		\begin{minipage}[c]{.46\linewidth}
			\centering
			\includegraphics[width=1\textwidth]{img/univ-montpellier.png}
		\end{minipage}
		\hfill%
		\begin{minipage}[c]{.46\linewidth}
			\centering
			\includegraphics[width=1\textwidth]{img/fds.png}
		\end{minipage}
	\end{figure}

	\par\vspace{1cm}

	\vfill

	% Bottom of the page
	{\large \today\par}
\end{titlepage}





% ------------------------------------- %
% Introduction
% ------------------------------------- %

\parskip=5pt



% Espacement entre les paragraphes
\parskip=5pt
% ------------------------------------- %
% Organisation
% ------------------------------------- %

\section{Structures de Données  }

\subsection{Structure Ressource : }
Une structure qui permet de stocker un entier représentant le nombre de CPU et un autre entier pour stocker les ressources de mémoire.
\subsection{Structure Site : }
Cette structure de données permet de stocker les informations d'un site donné (une ville) :

\begin{itemize}
	\item Une chaîne de caractères pour stocker le nom de la ville ou du site.
	\item Une instance de la structure ressource qui représente les ressources disponibles dans ce site (CPU, memory).
\end{itemize}
\subsection{Structure Cloud : }
Une structure qui représente un ensemble de sites.(le système de réservations).
\subsection{Structure Commande : }
Une structure représentant une commande saisie par l'utilisateur : 
\begin{itemize}
	\item Un type cmd\_t prédéfini permettant de stocker le type de la commande.
	\item Deux tableaux d'entiers pour stocker les entiers saisis de différents sites.
	\item Un tableaux de chaînes de caractères pour stocker les noms des sites saisis.
	\item Un entier qui permet de stocker le nombre de sites saisis.
\end{itemize}



\section{Les objest IPC  }
\subsection{Côté Serveur  }
\subsubsection{Segment de mémoire partagé : }
\begin{itemize}
	\item Une mémoire partagée entre les processus du serveur permettant de stocker sous la forme d'une chaîne de caractères le cloud lu à partir d'un fichier.
	\item un segment de mémoire partagé (un entier) pour stocker le nombre de threads concurrents du serveur et ses fils.
	\item un segment de mémoire partagé (un entier) pour stocker le nombre de threads qui sont arrivés au point de rendez vous.
\end{itemize}


\subsubsection{Les sémaphores :}
Un tableau constitué de quatre sémaphores : 
\begin{itemize}
	\item Un premier sémaphore protégeant la mémoire partagée des ressources des accès concurrents (un verrou).
	\item Un deuxième pour envoyer un signal lors d'une modification de l'état du système de réservations (le Cloud).
	\item Un troisième protégeant l'accès aux deux mémoires partagées qui permettent de mettre en oeuvre un système de rendez vous entre threads.
	\item Le dernier pour envoyer un signal lors que tous les threads sont arrivés au point de rendez vous.
\end{itemize}



\subsection{Côté Client  }
\subsubsection{Les sémaphores :}
Un tableau de sémaphores contenant un seul sémaphore, son rôle est de mettre le client en attente lorsque son serveur est en attente et le déverrouiller dès que sa demande soit satisfaite.

\section{Les Protocoles d'échange}

\subsection{Entre le client et le serveur}

Le client construit,suite à la saisie de l'utilisateur, une structure \textbf{commande} et l'envoi au serveur en utilisant la fonction \textbf{sendCommandeTCP()} et le serveur la reçois en utilisant la fonction \textbf{recvCommandeTCP()}. Les fonctions fonctionnent ainsi :
\begin{itemize}
	\item le type de la commande est d'abord envoyer sous forme d\'entier.
	\item puis viens le nombre de serveurs qu'il y a dans la commande.
	\item Enfin, à la suite, le nom du serveur, le nombre de CPU et de mémoire demander sont envoyés.
\end{itemize}

\subsection{Entre le serveur et le client}

\subsubsection{Entre le thread principal du serveur et le thread du client}

Le thread principal du serveur reste en écoute des demandes du client. Une fois la commande reçue (\textbf{recvCommandeTCP()}). Le serveur traite la demande et signal toute modification aux autres serveur. Puis envoient la réponse au thread du client qui la reçoit et l'affiche.

\subsubsection{Entre le thread du serveur et le thread du client}

Lorscequ'un signal de modification de l'état de la ressource est reçu, le thread envoie le nouvel état au thread de son client en utilisant la fonction \textbf{sendTCP()}. Cette dernière fonctionne ainsi :
\begin{itemize}
	\item La taille du message est envoyé.
	\item puis le message lui mếme.
\end{itemize}

\section{Processus et threads }

\subsection{Côté Serveur  }
\subsubsection{Le Serveur Principal et son threads : }
Le serveur principal se charge d'initialiser les objets IPC et de créer son thread qui se charge de récuperer à chaque modification le nouvel état du système pour que le serveur principal soit toujours à jour.
\subsubsection{Le Serveur Fils et son threads : }
Le serveur fils se charge de créer en premier son thread et ensuite attendre des commandes de la part de son client dédié, il vérifie et exécute la commande envoyée par son client, ensuite en envoyant un signal via un sémaphore déclenche son thread (\textbf{wait\_thread}) et les threads de tous les serveurs fils (qui sont en attente d'un signal), pour qu'à leurs tour, ils envoient le nouvel état du cloud à leurs du client(chaque thread d'un serveur fils envoie le nouvel état du cloud à son client dédié).



\subsection{Côté Client  }
Le thread principal demande une saisie d'une commande à l'utilisateur pour l'envoyé à son serveur dédié. 
Le thread \textbf{listen\_modif()} en boucle attends de recevoir un message de la part du thread de son serveur dédié ou le serveur dédié, que ça le cloud ou la réponse à une demande du thread principal.
Ce thread affiche chaque message reçu sur la sortie standard.




\end{document}
